\documentclass{article}

\usepackage{algorithm}

\usepackage[noend]{algpseudocode}
\renewcommand\algorithmicthen{}
\renewcommand\algorithmicdo{}
% New definitions
\algnewcommand\algorithmicswitch{\textbf{switch}}
\algnewcommand\algorithmiccase{\textbf{case}}
\algnewcommand\algorithmicassert{\texttt{assert}}
\algnewcommand\Assert[1]{\State \algorithmicassert(#1)}%
% New "environments"
\algdef{SE}[SWITCH]{Switch}{EndSwitch}[1]{\algorithmicswitch\ #1\ \algorithmicdo}{\algorithmicend\
\algorithmicswitch}%
\algdef{SE}[CASE]{Case}{EndCase}[1]{\algorithmiccase\ #1}{\algorithmicend\
\algorithmiccase}%
\algtext*{EndSwitch}%
\algtext*{EndCase}%

\usepackage{amsfonts}

\usepackage{amsmath}

\usepackage{amssymb}

\usepackage{amsfonts}

\usepackage{amsthm}
\theoremstyle{example}
\newtheorem{example}{Example}

\theoremstyle{remark}
\newtheorem{remark}{Remark}

\usepackage{bm}

\usepackage{booktabs}

\usepackage[justification=centering]{caption}

\usepackage{graphicx}

\usepackage[hidelinks]{hyperref}

\usepackage{mleftright}
\mleftright{}

\usepackage{tikz}

\DeclareFontFamily{U}{matha}{\hyphenchar\font45}
\DeclareFontShape{U}{matha}{m}{n}{
      <5> <6> <7> <8> <9> <10> gen * matha
      <10.95> matha10 <12> <14.4> <17.28> <20.74> <24.88> matha12
      }{}

\newcommand{\Land}{\mathbin{
  \raisebox{.1ex}{%
    \rotatebox[origin=c]{-90}{\usefont{U}{matha}{m}{n}\symbol{\string"CE}}}}}
\newcommand{\Lor}{\mathbin{
  \raisebox{.1ex}{%
    \rotatebox[origin=c]{90}{\usefont{U}{matha}{m}{n}\symbol{\string"CE}}}}}

\renewcommand{\implies}{\Rightarrow}
\newcommand{\Implies}{\Longrightarrow}
\renewcommand{\iff}{\Leftrightarrow}
\newcommand{\lxor}{\veebar}
\newcommand{\Iff}{\Longleftrightarrow}
\newcommand{\nmodels}{\not\models}
\newcommand{\BigO}{\mathcal{O}}
\newcommand{\abs}[1]{\left\lvert{#1}\right\rvert}

\author{Stefano Trevisani}
\date{\today}
\title{Notes for Verification and Validation Techniques for Artificial Intelligence and Cybersecurity}

\begin{document}
\maketitle
\clearpage
\tableofcontents
\clearpage

\section{Introduction to Logic}
Logic is a formal language to reason about the properties of some `objects'.
A logic, as any formal language, must have:
\begin{itemize}
	\item \emph{Lexicon} (or \emph{vocabulary}): symbols to denote objects and operations.
	\item \emph{Syntax}: rules to combine symbols into constructs (terms, formulae).
	\item \emph{Semantic}: meaning assigned to the syntactic constructs.
	\item \emph{Algorithms} (or \emph{operations}): transformations for reasoning about the semantic
	      of a syntactic construct.
\end{itemize}

\noindent A \emph{tautology} (or \emph{valid formula}) is a formula which is true in every model.
\begin{example}
	Consider the following proposition:
	\begin{quote}
		All men are mortal. Socrates is a man. Therefore, Socrates is mortal.
	\end{quote}
	It is a specific case of the more generic tautology:
	\[ \left(\forall x\colon A\left(x\right) \implies B\left(x\right)\right) \land A\left(y\right)
		\implies B\left(y\right) \]
\end{example}

\noindent A \emph{contradiction} (or \emph{paradox}) is a formula which is false in every model.
\begin{example}
	Consider the following proposition:
	\begin{quote}
		There is a barber which shaves everyone who doesen't shave themselves.
	\end{quote}
	It is a specific case of the more generic paradox:
	\[\exists x,\forall y\colon A\left(x,y\right) \iff \neg A\left(y,y\right)\]
\end{example}

\noindent The variables in a formula can be of different \emph{sort}, as they represent objects of
different \emph{type}.
\begin{example}
	Consider the following proposition:
	\begin{quote}
		Every incoming order is eventually processed
	\end{quote}
	It can be encoded in the following formula:
	\[\forall o \in O, \forall t \in T\colon arrives\left(o,t\right) \implies \exists t' \in T\colon
		t' > t \land process\left(o, t'\right)\]
\end{example}

\subsection{Problems on logic}
There are several problems which can be defined over a logic:

\begin{itemize}
	\item \emph{Evaluation} (or \emph{Model checking}, \textsc{eval}): given a formula
	      \(\phi \) and an interpretation \(M\), decide whether \(M \models \phi \)
	      (i.e.\ whether \(M\) is a model of \(\phi \)).
	\item \emph{Satisfiability} (\textsc{sat}): given a formula \(\phi \), decide whether \(\exists
	      M\colon M \models
	      \phi \).
	\item \emph{Validity} (\textsc{val}): given a formula \(\phi \), decide whether \(\forall M\colon M
	      \models \phi \).
	\item \emph{Equivalence} (\textsc{equiv}): given two formulae \(\phi, \psi \), decide whether
	      \(\phi \equiv \psi \).
	\item \emph{Consequence} (\textsc{cons}): given two formulae \(\phi, \psi \), decide
	      whether \(\phi \Implies \psi \).
\end{itemize}

Different logics can have similar problems associated with them, but they often differ in
\emph{expressive power} and \emph{succintness}: problems associated with more expressive logics
belong to higher complexity classes, and can even be undecidable, while succint logics offer
more compact representations of similar problems, (artificially) increasing the associated complexity.

\subsection{Propositional logic}
Propositional logic is the simplest kind of formal logic:
\begin{itemize}
	\item Its \emph{lexicon} consists of propositional symbols, tipically denoted with lowercase
	      letters \(p, q, r, \dots \) and the boolean connectives \(\neg, \land, \lor, \implies, \iff \).
	      The set of all propositional symbols (the \emph{alphabet}) is denoted with \(\Sigma \).
	\item Its \emph{syntax} consists of formulae denoted with greek letters \(\phi, \psi, \dots \)
	      such that:
	      \[\phi := {p} \mid {\neg \psi} \mid {\psi \land \psi'} \mid {\psi \lor \psi'} \mid
		      {\psi \implies \psi'} \mid {\psi \iff \psi'} \]
	\item Its \emph{semantic} consists of an \emph{interpretation} function
	      \(M\colon \Sigma \mapsto \left\{\bot, \top \right\} \), which \emph{holds} on (or
	      \emph{models}, or
	      \emph{entails}) a formula \(\phi \):
	      \begin{align*}
		       & M \models {p}                  &  & \Iff &  & M\left(p\right) = \top
		      \\
		       & M \models {\neg \phi}          &  & \Iff &  & M \nmodels {\phi}
		      \\
		       & M \models {\phi \land \psi}    &  & \Iff &  & {M \models \phi} \Land {M \models
			      \psi}
		      \\
		       & M \models {\phi \lor \psi}     &  & \Iff &  & {M \models \phi} \Lor {M \models \psi}
		      \\
		       & M \models {\phi \implies \psi} &  & \Iff &  & {M \models \phi} \Implies {M \models
			      \psi}
		      \\
		       & M \models {\phi \iff \psi}     &  & \Iff &  & {M \models \phi} \Iff {M \models \psi}
	      \end{align*}

	      \noindent Furthermore:
	      \begin{align*}
		       & \phi \textnormal{ is a \emph{tautology}}            &  & \Iff &  & \forall M\colon M
		      \models \phi
		      \\
		       & \phi \textnormal{ is a \emph{contradiction}}        &  & \Iff &  & \forall M\colon M
		      \nmodels \phi
		      \\
		       & \phi \textnormal{ is a \emph{consequence} of } \psi &  & \Iff &  & \forall
		      M\colon M \models \psi \Implies M \models \phi
		      \\
		       & \phi \textnormal{ is \emph{equivalent} to } \psi    &  & \Iff &  &
		      \forall
		      M\colon M \models \phi \Iff M \models \psi
	      \end{align*}
\end{itemize}

\begin{remark}
	Given two formulae \(\phi \) and \(\psi \) and the proposition \(\phi \implies \psi \):
	\begin{itemize}
		\item \(\phi \implies \psi \) is the \emph{direct} proposition.
		\item \(\psi \implies \phi \) is the \emph{converse} proposition.
		\item \(\neg\phi \implies \neg\psi \) is the \emph{inverse} proposition.
		\item \(\neg\psi \implies \neg\phi \) is the \emph{contrapositive} proposition.
		\item If the direct is false, the inverse is true.
		\item If the direct is true, the contrapositive is true.
	\end{itemize}
\end{remark}

\noindent To solve the model checking problem for propositional logic, we can use
Algorithm~\ref{alg:prop_model_check}, which is a textbook recursive divide-and-conquer algorithm
taking \(\BigO\left(\abs{\phi}\right)\) time (assuming that evaluating \(M\left(p\right)\) takes constant
time).
\begin{algorithm}
	\centering
	\begin{algorithmic}
		\Require{A NNF formula \(\phi \) and an interpretation \(M\)}
		\Ensure{Whether \(M\) is a model for \(\phi \)}
		\Function{modelCheck}{$\phi$, $M$}
		\Switch{\(\phi \)}
		\Case{\(p\)}
		\If{\(M\left(p\right) = \top \)}
		\State{\Return{\(\textbf{true}\)}}
		\Else{}
		\State{\Return{\(\textbf{false}\)}}
		\EndIf{}
		\EndCase{}

		\Case{\(\neg \phi_1\)}
		\State{\Return{\(\neg \Call{modelCheck}{\phi_1, M}\)}}
		\EndCase{}

		\Case{\(\phi_1 \land \phi_2\)}
		\State{\Return{\(\Call{modelCheck}{\phi_1, M}\land \Call{modelCheck}{\phi_2, M}\)}}
		\EndCase{}

		\Case{\(\phi_1 \lor \phi_2\)}
		\State{\Return{\(\Call{modelCheck}{\phi_1, M} \lor \Call{modelCheck}{\phi_2, M}\)}}
		\EndCase{}
		\EndSwitch{}
		\EndFunction{}
	\end{algorithmic}
	\caption{An algorithm to evaluate a propositional formula}\label{alg:prop_model_check}
\end{algorithm}

\noindent It is well known that \textsc{eval} is a \textbf{P}-complete problem\footnote{
	\(\textbf{P} = TIME\left(\BigO\left(N^k\right)\right)\) is the class of problems decidable by a
	deterministic Turing machine in time at most polynomial to the input size \(N\).}, and that it
is possible to convert any propositional formula to an equivalent NNF formula in polynomial time.

On the other hand, to solve the \textsc{sat} problem for propositional logic, we can use
Algorithm~\ref{alg:prop_sat}, which takes exponential time to execute.
\begin{algorithm}
	\centering
	\begin{algorithmic}
		\Require{A NNF formula \(\phi \)}
		\Ensure{Whether \(\phi \) is satisfiable or not}
		\Function{isSatisfiable}{$\phi$}
		\State{\(M \gets {\left(\bot\right)}^{\abs{\phi}}\)}
		\While{\(M \neq {\left(\top\right)}^{\abs{\phi}}\)}
		\If{\(\Call{modelCheck}{\phi, M} = \textbf{true}\)}
		\State{\Return{\(\textbf{true}\)}}
		\EndIf{}
		\State{\(M \gets \Call{nextCombination}{M}\)}
		\EndWhile{}
		\If{\(\Call{modelCheck}{\phi, M} = \textbf{true}\)}
		\State{\Return{\(\textbf{true}\)}}
		\EndIf{}
		\State{\Return{\(\textbf{false}\)}}
		\EndFunction{}
	\end{algorithmic}
	\caption{An algorithm to decide satisfiability of a propositional formula}\label{alg:prop_sat}
\end{algorithm}

\noindent It is well known that \textsc{sat} for propositional logic is an \textbf{NP}-complete
problem\footnote{\(\textbf{NP} = NTIME\left(\BigO\left(N^k\right)\right)\) is the class of problems decidable
	by a non-deterministic Turing machine in time at most polynomial to the input size \(N\).}.

\subsubsection{Normal forms}
All propositional formulae can be rewritten in one of the so-called \emph{normal forms}, which
reduce the vocabulary and constrain the syntactic structure of formulae with the advantage of
making them easier to work on.
There are three main normal forms:
\begin{itemize}
	\item \emph{Negation Normal Form} (NNF): the only allowed connectors are \(\neg \), \(\land \) and
	      \(\lor \). Since \(x \implies y \equiv \neg x \lor y\) and
	      \(x \iff y \equiv x \implies y \land y \implies x\), we do not lose expressive power.
	      There are no restrictions on the syntactic structure of a formula.
	      If a formula \(\phi \) contains two-way implications, then \(\abs{NNF\left(\phi\right)} =
	      \BigO\left(2^{\abs{\phi}}\right)\),
	      otherwise it will always be \(\BigO\left(\abs{\phi}\right)\).
	\item \emph{Conjunctive Normal Form} (CNF): the allowed connectors are the same of NNF, but
	      all the conjuctions must be at the top level of a formula:
	      \[\phi = \left(l_{1,1} \lor \dots \lor l_{1,n_1}\right) \land \dots \land \left(l_{m,1}
		      \lor \dots \lor l_{m,n_m}\right)\]
	      where \(l = p \mid \neg p\).
	      It is always possible to translate any formula in an equivalent CNF formula using
	      De Morgan's laws and distributive laws:
	      \begin{itemize}
		      \renewcommand\labelitemii{\(\circ \)}
		      \item \(x \lor y \equiv \neg{\left(\neg x \land \neg y\right)}\)
		      \item \(x \land y \equiv \neg{\left(\neg x \lor \neg y\right)}\)
		      \item \(x \lor \left(y \land z\right) = \left(x \lor y\right) \land \left(x \lor
		            z\right)\)
		      \item \(x \land \left(y \lor z\right) = \left(x \land y\right) \lor \left(x \land
		            z\right)\)
	      \end{itemize}
	      Unfortunately, \(\abs{CNF\left(\phi\right)} = \BigO\left(2^{\abs{\phi}}\right)\) in general,
	      but it is possible to introduce new additional variables to keep the size linear: we lose
	      \emph{equivalence} but we keep \emph{equisatisfiability} (\textsc{CNF-sat} is still
	      \textbf{NP}-complete).
	      This form is also known as \emph{Product of Sums} (POS) form.
	\item \emph{Disjunctive Normal Form} (DNF): the allowed connectors are the same of NNF, but
	      all the disjunctions must be at the top level of a formula:
	      \[\phi = \left(l_{1,1} \land \dots \land l_{1,n_1}\right) \lor \dots \lor \left(l_{m,1}
		      \land \dots \land l_{m,n_m}\right)\]
	      where \(l = p \mid \neg p\).
	      It is always possible to translate any formula in an equivalent DNF formula using
	      De Morgan's laws and distributive laws.
	      In general, \(\abs{DNF\left(\phi\right)} = \BigO\left(2^{\abs{\phi}}\right)\),
	      and there is no known satisfiability-preserving polynomial space transformation
	      (in fact, \textsc{DNF-sat} is in \textbf{P}).
	      This form is also known as \emph{Sum of Products} (SOP) form.
\end{itemize}

\subsubsection{Tableaux method}
To solve the satisfiability problem for propositional logic, we saw that we could use
Algorithm~\ref{alg:prop_sat}, assuming the formula was in NNF\@.
A better approach to the \textsc{sat} problem is the \emph{tableaux method}, which is a special
case of the backtracking method.
More generally, backtracking algorithms are devised to solve \emph{Constraint Satisfaction Problems}
(CSPs), i.e.\ search problems involving a  set variables \(V = \left\{x_1, \dots, x_n \right\} \)
each over some domain \(\mathcal{D} = \left\{D_1, \dots, D_n \right\} \) subject to a set of
constraints \(C\).
For propositional logic, \(D_1 = \cdots = D_n = \left\{\bot, \top \right\} \), and the constraints
are determined by the logical connectors between the subformulae.
If we a formula \(\phi \) left-to-right, we can see the search space as a binary tree where at each
level \(k\) we split the tree in two branches if \(\phi_k = \phi_{k+1,l} \lor \phi_{k+1,r} \) (we
assume the left side to be true in the left branch, and the right side to be true in the right branch)
or we split the formula if \(\phi_k = \phi_{k+1,l} \land \phi_{k+1,r} \)
(as both sides must be true).
Every leaf of the tree will then represent an interpretation, which becomes a model if the
formula is satisfied.
While the tree can contain up to \(2^n\) leaves, this does not always happen, as partial assignments
might cause an early invalidation of the formula.
The asymptotical complexity is still exponential, of course, but backtracking and tableaux methods
can exploit many kinds of heuristics which often drastically reduce the search time.

\subsubsection{Transition systems}
Consider a \(k\)-bit finite state machine \(\mathcal{M}\), which can be in one of \(n = 2^k\)
possible states \(S_i = \left( \left\lceil{\frac{i}{2^{k-1}}}\right\rceil \bmod 2, \dots,
\left\lceil{\frac{i}{2^{0}}}\right\rceil \bmod 2\right)\).
We can describe the \emph{reachability} problem \textsc{reach} as follows:
\begin{itemize}
	\item A formula \(\phi_{init}\left(X\right)\) which describes whether a state can be the initial
	      state of the system.
	\item A formula \(\phi_{goal}\left(X\right)\) which describes whether a state can be the final
	      target of the system.
	\item A formula \(\phi_{tran}\left(X, Y\right)\) which describes if it is possible for the system
	      to transition from the first state to the second state.
	\item A formula \(\phi_{same}\left(X, Y\right) = \bigwedge_{i=0}^{k}\left(X_i \iff Y_i\right)\),
	      which describes whether two states are the same.
	      It is not needed if \(\phi_{trans}\left(X,X\right)\) is always true.
\end{itemize}
Then, the formula which describes the reachability problem from an initial state \(X_1\) to a
final state \(X_n\) passing by states \(X_2, \dots, X_{n-1}\) is simply:
\[
	\phi(X_1, \dots, X_n) = \phi_{init}\left(X_1\right) \wedge \phi_{goal}\left(X_n\right) \wedge
	\bigwedge_{i=1}^{n}{\phi_{tran}\left(X_i, X_{i+1}\right) \lor \phi_{stay}(X_i, X_{i+1})}
\]
Note that, in \(\phi \), we didn't just fix the initial and final states, but also all the
intermediate states: if there is an alternative path to reach the final state, it will not be
taken into account from our formula.
Furthermore, if \(\mathcal{M}\) cannot get from a state \(A\) to a state \(B\) in less than
\(2^k\) steps, then there is no way to get from \(A\) to \(B\).

\begin{example}
	Suppose we have a 4-bit finite state machine \(\mathcal{M}\), then we have \(2^4 = 16\) different
	possible states.
	We define \(\phi_{init}(X)\) as:
	\[\phi_{init}\left(X\right) = \neg X_2 \land \neg X_3 \land \neg X_4 \]
	The only states which satisfy \(\phi_{init}\) are \(S_0 = \left(0, 0, 0, 0\right)\) and
	\(S_1 = \left(1, 0, 0, 0\right)\).
	Then we define \(\phi_{goal}\) as:
	\[\phi_{goal}\left(X\right) = X_1 \land X_2 \land X_3 \land X_4\]
	Which is satisfied only by \(S_{15} = \left(1, 1, 1, 1\right)\)\footnote{Since
		\(\left\{\bot, \top\right\} \) is isomorphic to \(\left\{0, 1\right\} \), we use the two sets
		interchangeably.}.

	Suppose that our machine is a counter, then the transition formula will be:
	%\begin{noindent}
	\begin{align*}
		& \phi_{tran}\left(X,Y\right) =                                                      \\
		& \left(\neg X_1 \lor \neg X_2 \lor \neg X_3 \lor \neg X_4 \lor \neg Y_1\right) \land
		\left(X_1 \lor \neg X_2 \lor \neg X_3 \lor \neg X_4 \lor Y_1\right) & \land						\\
		%
		& \left(\neg X_1 \lor Y_1 \lor \neg Y_2\right) \land 
		\left(X_1 \lor \neg Y_1 \lor \neg Y_2\right) \land
		\left(\neg X_1 \lor X_2 \lor Y_1\right) \land
		\left(X_1 \lor X_2 \lor \neg Y_1\right)                             & \land						\\
		%
		& \left(\neg X_2 \lor Y_2 \lor \neg Y_3\right) \land
		\left(X_2 \lor \neg Y_2 \lor \neg Y_3\right) \land
		\left(\neg X_2 \lor Y_2 \lor \neg Y_4\right) \land
		\left(X_2 \lor \neg Y_2 \lor \neg Y_4\right)                        & \land						\\
		% 
		& \left(\neg X_3 \lor Y_3 \lor \neg Y_4\right) \land
		\left(X_3 \lor \neg Y_3 \lor \neg Y_4\right) \land
		\left(X_3 \lor \neg X_4 \lor Y_3\right)                             & \land						\\
		& \left(X_2 \lor \neg X_3 \lor \neg X_4 \lor Y_2\right) \land
		\left(X_4 \lor Y_4\right)
	\end{align*}
	%\end{noindent}

	\noindent This formula is satisfied only when applied to \(S_{i}\) and \(S_{i+1 \bmod 16}\), as
	it states that the 4-bit number \(Y\) is the successor (modulo 16) of the 4-bit numer \(X\).
	Now, let's check some cases when trying to satisfy \(\phi\left(X_1, \dots, X_n\right)\):
	\begin{itemize}
		\item If we choose \(X_i = S_i\), then \(\phi \) is satisfied, as \(S_0\) is
		      a valid initial state (\(\phi_{init}\) is satisfied), \(S_n\) is a valid final state
		      (\(\phi_{goal}\) is satisfied), and it is possible to transition from \(S_i\) to
		      \(S_{i+1}\), (all occurences of \(\phi_{tran}\) are satisfied).
		\item If \(X_{15} \neq S_{15}\), \(\phi_{goal}\) won't be satisfied,
		      as \(S_{15}\) is the only valid final state.
		\item Similarly, \(\phi_{init}\) will never be satisfied if \(X_1\) is neither \(S_1\)
		      nor \(S_8\).
		\item Again, some \(\phi_{tran}\) won't be satisfied if, when \(X_{i} = S_{j}\),
		      \(X_{i+1} \neq S_{j+1}\).
		      However, if \(X_{i+1} = X_{i}\), then \(\phi_{stay}\) is still satisfied.
		\item Assume that \(X_1 = S_8\) and \(X_{15} = S_{15}\), it is possible to transition from 
					\(S_8\) to \(S_{15}\) in \(7 < 15\) steps.
		      Then, we set \(X_2 = S_9, \dots, X_7 = S_{14}\) and \(X_8 = \cdots = X_{15}\):
		      thanks to \(\phi_{stay}\), the whole formula is satisfied.
	\end{itemize}
\end{example}

\subsection{Quantified propositional logic}
\end{document}
